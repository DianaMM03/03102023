% Options for packages loaded elsewhere
\PassOptionsToPackage{unicode}{hyperref}
\PassOptionsToPackage{hyphens}{url}
%
\documentclass[
]{article}
\usepackage{amsmath,amssymb}
\usepackage{iftex}
\ifPDFTeX
  \usepackage[T1]{fontenc}
  \usepackage[utf8]{inputenc}
  \usepackage{textcomp} % provide euro and other symbols
\else % if luatex or xetex
  \usepackage{unicode-math} % this also loads fontspec
  \defaultfontfeatures{Scale=MatchLowercase}
  \defaultfontfeatures[\rmfamily]{Ligatures=TeX,Scale=1}
\fi
\usepackage{lmodern}
\ifPDFTeX\else
  % xetex/luatex font selection
\fi
% Use upquote if available, for straight quotes in verbatim environments
\IfFileExists{upquote.sty}{\usepackage{upquote}}{}
\IfFileExists{microtype.sty}{% use microtype if available
  \usepackage[]{microtype}
  \UseMicrotypeSet[protrusion]{basicmath} % disable protrusion for tt fonts
}{}
\makeatletter
\@ifundefined{KOMAClassName}{% if non-KOMA class
  \IfFileExists{parskip.sty}{%
    \usepackage{parskip}
  }{% else
    \setlength{\parindent}{0pt}
    \setlength{\parskip}{6pt plus 2pt minus 1pt}}
}{% if KOMA class
  \KOMAoptions{parskip=half}}
\makeatother
\usepackage{xcolor}
\usepackage[margin=1in]{geometry}
\usepackage{color}
\usepackage{fancyvrb}
\newcommand{\VerbBar}{|}
\newcommand{\VERB}{\Verb[commandchars=\\\{\}]}
\DefineVerbatimEnvironment{Highlighting}{Verbatim}{commandchars=\\\{\}}
% Add ',fontsize=\small' for more characters per line
\usepackage{framed}
\definecolor{shadecolor}{RGB}{248,248,248}
\newenvironment{Shaded}{\begin{snugshade}}{\end{snugshade}}
\newcommand{\AlertTok}[1]{\textcolor[rgb]{0.94,0.16,0.16}{#1}}
\newcommand{\AnnotationTok}[1]{\textcolor[rgb]{0.56,0.35,0.01}{\textbf{\textit{#1}}}}
\newcommand{\AttributeTok}[1]{\textcolor[rgb]{0.13,0.29,0.53}{#1}}
\newcommand{\BaseNTok}[1]{\textcolor[rgb]{0.00,0.00,0.81}{#1}}
\newcommand{\BuiltInTok}[1]{#1}
\newcommand{\CharTok}[1]{\textcolor[rgb]{0.31,0.60,0.02}{#1}}
\newcommand{\CommentTok}[1]{\textcolor[rgb]{0.56,0.35,0.01}{\textit{#1}}}
\newcommand{\CommentVarTok}[1]{\textcolor[rgb]{0.56,0.35,0.01}{\textbf{\textit{#1}}}}
\newcommand{\ConstantTok}[1]{\textcolor[rgb]{0.56,0.35,0.01}{#1}}
\newcommand{\ControlFlowTok}[1]{\textcolor[rgb]{0.13,0.29,0.53}{\textbf{#1}}}
\newcommand{\DataTypeTok}[1]{\textcolor[rgb]{0.13,0.29,0.53}{#1}}
\newcommand{\DecValTok}[1]{\textcolor[rgb]{0.00,0.00,0.81}{#1}}
\newcommand{\DocumentationTok}[1]{\textcolor[rgb]{0.56,0.35,0.01}{\textbf{\textit{#1}}}}
\newcommand{\ErrorTok}[1]{\textcolor[rgb]{0.64,0.00,0.00}{\textbf{#1}}}
\newcommand{\ExtensionTok}[1]{#1}
\newcommand{\FloatTok}[1]{\textcolor[rgb]{0.00,0.00,0.81}{#1}}
\newcommand{\FunctionTok}[1]{\textcolor[rgb]{0.13,0.29,0.53}{\textbf{#1}}}
\newcommand{\ImportTok}[1]{#1}
\newcommand{\InformationTok}[1]{\textcolor[rgb]{0.56,0.35,0.01}{\textbf{\textit{#1}}}}
\newcommand{\KeywordTok}[1]{\textcolor[rgb]{0.13,0.29,0.53}{\textbf{#1}}}
\newcommand{\NormalTok}[1]{#1}
\newcommand{\OperatorTok}[1]{\textcolor[rgb]{0.81,0.36,0.00}{\textbf{#1}}}
\newcommand{\OtherTok}[1]{\textcolor[rgb]{0.56,0.35,0.01}{#1}}
\newcommand{\PreprocessorTok}[1]{\textcolor[rgb]{0.56,0.35,0.01}{\textit{#1}}}
\newcommand{\RegionMarkerTok}[1]{#1}
\newcommand{\SpecialCharTok}[1]{\textcolor[rgb]{0.81,0.36,0.00}{\textbf{#1}}}
\newcommand{\SpecialStringTok}[1]{\textcolor[rgb]{0.31,0.60,0.02}{#1}}
\newcommand{\StringTok}[1]{\textcolor[rgb]{0.31,0.60,0.02}{#1}}
\newcommand{\VariableTok}[1]{\textcolor[rgb]{0.00,0.00,0.00}{#1}}
\newcommand{\VerbatimStringTok}[1]{\textcolor[rgb]{0.31,0.60,0.02}{#1}}
\newcommand{\WarningTok}[1]{\textcolor[rgb]{0.56,0.35,0.01}{\textbf{\textit{#1}}}}
\usepackage{graphicx}
\makeatletter
\def\maxwidth{\ifdim\Gin@nat@width>\linewidth\linewidth\else\Gin@nat@width\fi}
\def\maxheight{\ifdim\Gin@nat@height>\textheight\textheight\else\Gin@nat@height\fi}
\makeatother
% Scale images if necessary, so that they will not overflow the page
% margins by default, and it is still possible to overwrite the defaults
% using explicit options in \includegraphics[width, height, ...]{}
\setkeys{Gin}{width=\maxwidth,height=\maxheight,keepaspectratio}
% Set default figure placement to htbp
\makeatletter
\def\fps@figure{htbp}
\makeatother
\setlength{\emergencystretch}{3em} % prevent overfull lines
\providecommand{\tightlist}{%
  \setlength{\itemsep}{0pt}\setlength{\parskip}{0pt}}
\setcounter{secnumdepth}{-\maxdimen} % remove section numbering
\ifLuaTeX
  \usepackage{selnolig}  % disable illegal ligatures
\fi
\IfFileExists{bookmark.sty}{\usepackage{bookmark}}{\usepackage{hyperref}}
\IfFileExists{xurl.sty}{\usepackage{xurl}}{} % add URL line breaks if available
\urlstyle{same}
\hypersetup{
  pdftitle={Example with maps},
  hidelinks,
  pdfcreator={LaTeX via pandoc}}

\title{Example with maps}
\author{true}
\date{2023-10-03}

\begin{document}
\maketitle

{
\setcounter{tocdepth}{6}
\tableofcontents
}
\hypertarget{interactive-maps-using-leaflet}{%
\subsection{Interactive maps using
leaflet}\label{interactive-maps-using-leaflet}}

Example extracted from
\href{https://roh.engineering/posts/2021/05/map-symbols-and-size-legends-for-leaflet/}{this
post}.

More information about leaflet in R
\href{http://rstudio.github.io/leaflet/}{here}.

\begin{Shaded}
\begin{Highlighting}[]
\FunctionTok{data}\NormalTok{(}\StringTok{"quakes"}\NormalTok{)}
\end{Highlighting}
\end{Shaded}

\begin{Shaded}
\begin{Highlighting}[]
\NormalTok{symbols }\OtherTok{\textless{}{-}} \FunctionTok{makeSymbolsSize}\NormalTok{(}
  \AttributeTok{values =}\NormalTok{ quakes}\SpecialCharTok{$}\NormalTok{depth,}
  \AttributeTok{shape =} \StringTok{\textquotesingle{}diamond\textquotesingle{}}\NormalTok{,}
  \AttributeTok{color =} \StringTok{\textquotesingle{}red\textquotesingle{}}\NormalTok{,}
  \AttributeTok{fillColor =} \StringTok{\textquotesingle{}red\textquotesingle{}}\NormalTok{,}
  \AttributeTok{opacity =}\NormalTok{ .}\DecValTok{5}\NormalTok{,}
  \AttributeTok{baseSize =} \DecValTok{5}
\NormalTok{)}
\FunctionTok{leaflet}\NormalTok{() }\SpecialCharTok{\%\textgreater{}\%}
  \FunctionTok{addTiles}\NormalTok{() }\SpecialCharTok{\%\textgreater{}\%}
  \FunctionTok{addMarkers}\NormalTok{(}\AttributeTok{data =}\NormalTok{ quakes,}
             \AttributeTok{icon =}\NormalTok{ symbols,}
             \AttributeTok{lat =} \SpecialCharTok{\textasciitilde{}}\NormalTok{lat, }\AttributeTok{lng =} \SpecialCharTok{\textasciitilde{}}\NormalTok{long) }\SpecialCharTok{\%\textgreater{}\%}
  \FunctionTok{addLegendSize}\NormalTok{(}
    \AttributeTok{values =}\NormalTok{ quakes}\SpecialCharTok{$}\NormalTok{depth,}
    \AttributeTok{color =} \StringTok{\textquotesingle{}red\textquotesingle{}}\NormalTok{,}
    \AttributeTok{fillColor =} \StringTok{\textquotesingle{}red\textquotesingle{}}\NormalTok{,}
    \AttributeTok{opacity =}\NormalTok{ .}\DecValTok{5}\NormalTok{,}
    \AttributeTok{title =} \StringTok{\textquotesingle{}Depth\textquotesingle{}}\NormalTok{,}
    \AttributeTok{shape =} \StringTok{\textquotesingle{}diamond\textquotesingle{}}\NormalTok{,}
    \AttributeTok{orientation =} \StringTok{\textquotesingle{}horizontal\textquotesingle{}}\NormalTok{,}
    \AttributeTok{position =} \StringTok{\textquotesingle{}bottomright\textquotesingle{}}\NormalTok{,}
    \AttributeTok{breaks =} \DecValTok{5}\NormalTok{)}
\end{Highlighting}
\end{Shaded}

\begin{verbatim}
## PhantomJS not found. You can install it with webshot::install_phantomjs(). If it is installed, please make sure the phantomjs executable can be found via the PATH variable.
\end{verbatim}

\includegraphics{05_Interactive-maps_files/figure-latex/unnamed-chunk-1-1.pdf}

\begin{Shaded}
\begin{Highlighting}[]
\NormalTok{numPal }\OtherTok{\textless{}{-}} \FunctionTok{colorNumeric}\NormalTok{(}\StringTok{\textquotesingle{}viridis\textquotesingle{}}\NormalTok{, }\DecValTok{10}\SpecialCharTok{\^{}}\NormalTok{(quakes}\SpecialCharTok{$}\NormalTok{mag))}
\FunctionTok{leaflet}\NormalTok{(quakes) }\SpecialCharTok{\%\textgreater{}\%}
  \FunctionTok{addTiles}\NormalTok{() }\SpecialCharTok{\%\textgreater{}\%}
  \FunctionTok{addSymbolsSize}\NormalTok{(}\AttributeTok{values =} \SpecialCharTok{\textasciitilde{}}\DecValTok{10}\SpecialCharTok{\^{}}\NormalTok{(mag),}
                 \AttributeTok{lat =} \SpecialCharTok{\textasciitilde{}}\NormalTok{lat, }
                 \AttributeTok{lng =} \SpecialCharTok{\textasciitilde{}}\NormalTok{long,}
                 \AttributeTok{shape =} \StringTok{\textquotesingle{}plus\textquotesingle{}}\NormalTok{,}
                 \AttributeTok{color =} \SpecialCharTok{\textasciitilde{}}\FunctionTok{numPal}\NormalTok{(}\DecValTok{10}\SpecialCharTok{\^{}}\NormalTok{(mag)),}
                 \AttributeTok{fillColor =} \SpecialCharTok{\textasciitilde{}}\FunctionTok{numPal}\NormalTok{(}\DecValTok{10}\SpecialCharTok{\^{}}\NormalTok{(mag)),}
                 \AttributeTok{opacity =}\NormalTok{ .}\DecValTok{5}\NormalTok{,}
                 \AttributeTok{baseSize =} \DecValTok{1}\NormalTok{) }\SpecialCharTok{\%\textgreater{}\%}
  \FunctionTok{addLegendSize}\NormalTok{(}
    \AttributeTok{values =} \SpecialCharTok{\textasciitilde{}}\DecValTok{10}\SpecialCharTok{\^{}}\NormalTok{(mag),}
    \AttributeTok{pal =}\NormalTok{ numPal,}
    \AttributeTok{title =} \StringTok{\textquotesingle{}Magnitude\textquotesingle{}}\NormalTok{,}
    \AttributeTok{baseSize =} \DecValTok{1}\NormalTok{,}
    \AttributeTok{shape =} \StringTok{\textquotesingle{}plus\textquotesingle{}}\NormalTok{,}
    \AttributeTok{orientation =} \StringTok{\textquotesingle{}horizontal\textquotesingle{}}\NormalTok{,}
    \AttributeTok{opacity =}\NormalTok{ .}\DecValTok{5}\NormalTok{,}
    \AttributeTok{fillOpacity =}\NormalTok{ .}\DecValTok{3}\NormalTok{,}
    \AttributeTok{position =} \StringTok{\textquotesingle{}bottomleft\textquotesingle{}}\NormalTok{,}
    \AttributeTok{breaks =} \DecValTok{5}\NormalTok{)}
\end{Highlighting}
\end{Shaded}

\includegraphics{05_Interactive-maps_files/figure-latex/unnamed-chunk-2-1.pdf}

\hypertarget{interactive-table-using-dt}{%
\subsection{Interactive table using
DT}\label{interactive-table-using-dt}}

See \href{https://rstudio.github.io/DT/}{this post} for more details
about doing tables with DT.

This table has all the data used in the previous maps.

\begin{Shaded}
\begin{Highlighting}[]
\FunctionTok{datatable}\NormalTok{(quakes)}
\end{Highlighting}
\end{Shaded}

\includegraphics{05_Interactive-maps_files/figure-latex/unnamed-chunk-3-1.pdf}

\end{document}
